\subsection{Spectral Analysis}
While machine learning will enable us to identify the scattering within large data
volumes associated with the structural response to electronic order, modeling the
data is an equally important data-driven challenge if we are to draw scientific
inferences from a large number of data sets. Most analyses of diffuse scattering
require a simulation of S(\textbf{Q}) using atomistic models that parametrize the
disorder, but these models can take weeks or months to refine. However, the
comprehensive \textbf{Q}-coverage in our data allows us to FT the data to generate
real-space PDFs in three dimensions. This has the advantage of converting complicated
intensity distributions in reciprocal space into discrete peaks in real space, whose
positions and intensities are given by the interatomic vectors present in the
disordered structure and their weighted probabilities, respectively.
Three-dimensional PDF measurements can provide much more information than the widely
used polycrystalline PDF method, but multidimensional signal processing is a
remarkably under-developed field. For example, our recent work on fluid flow analysis
was the first to determine truly multi-dimensional optimal data tapers in general
domains \cite{geoga2018}. In this proposal, our main goal is to correct this
deficiency in higher-dimensional spectral analysis and remove some of the artifacts
that one-dimensional-based methods cannot cope with. Our research will extend
one-dimensional \emph{multitaper spectrum estimation} methods to higher dimensional
Cartesian data, as well as utilize advanced tools such as \emph{cepstrum analysis}.
We will fuse the ML results with signal processing by converting
temperature-dependent structures isolated in regions of reciprocal space to their
real-space analogues, carefully conserving power with an optimal tapering technique.
By enabling physical interpretations of highly complex data sets that are both
intuitive and robust, these techniques will revolutionize the analysis of single
crystal total scattering. 

\subsubsection{3D-$\Delta$PDF Method}

\begin{figure}[!b]
\begin{center}
\includegraphics[width=0.85\textwidth]{FIGS/V2O5-PDF-model.pdf}
\caption{\label{fig:V2O5-PDF-model} (a) Real space model of sodium correlations in
Na$_{0.45}$V$_2$O$_5$ in the $x=0$ plane compared to (b) the $\Delta$PDF transform,
measured with high-energy x-ray scattering \cite{Krogstad:2019tc}. The sodium sites
form two-leg ladders and the zig-zag model in (a) is derived by connecting occupied
sites with more probable (red) vectors, ignoring neighboring sites with less probable
(blue) vectors. Three representative interatomic vectors connecting occupied sites
are shown (A, B, C).}
\end{center}
\end{figure}

In single crystals, the FT can be constrained to include only those interatomic
vectors whose probabilities deviate from the average structure, a major
simplification that is not possible in one-dimensional PDF analysis. This is achieved
by the ``punch-and-fill'' method pioneered by Weber \emph{et al} \cite{Weber:2012en},
which utilizes the fact that the total scattering can be separated into two
components: one representing the average crystal, \textit{i.e.}, the Bragg peaks, and
the other representing the diffuse scattering from crystalline defects. The FT of the
total scattering produces a Patterson function, \textit{i.e.}, the auto-correlation
function of the scattering, which can also be separated.
\begin{equation}
\begin{split}
P_{tot}(\boldsymbol{\mathrm{r}})&=\mathrm{FT}[|F_{H\! K\!
L}(\boldsymbol{\mathrm{Q}})|^2]+\mathrm{FT}[|\Delta F(\boldsymbol{\mathrm{Q}})|^2] \\
&=P_{H\! K\! L}(\boldsymbol{\mathrm{r}})+\Delta P(\boldsymbol{\mathrm{r}})
\end{split}
\end{equation}
where $F(\boldsymbol{\mathrm{Q}})$ is proportional to the FT of the electron density.
$P_{H\! K\! L}$ is the Patterson function from the average structure, while $\Delta
P(\boldsymbol{\mathrm{r}})$, or the 3D-$\Delta$PDF, is the difference Patterson
function due to the disorder. The ``punch-and-fill'' method isolates $\Delta
P(\boldsymbol{\mathrm{r}})$ by removing the scattering in a small sphere around each
Bragg peak and interpolating over the missing data before performing the FT. 

\begin{figure}[!t]
\begin{center}
\includegraphics[width=\textwidth]{FIGS/NaxV2O5-summary.pdf}
\caption{\label{fig:NaxV2O5-summary} Short-range sodium-sodium correlations in
Na$_{0.45}$V$_2$O$_5$, measured with high-energy x-ray scattering
\cite{Krogstad:2019tc} (a) $\Delta$PDF in the real-space y-z plane at 200K showing
the correlations extending to 100~\AA\ ($b=3.61$~\AA, $c=10.05$~\AA). This is an
extended version of the $\Delta$PDF in Fig. \ref{fig:V2O5-PDF-model}b. (b)
Temperature dependence of the ionic correlations along the z-axis fitted to an
exponential decay. (c) Temperature dependence of the correlation lengths along the x
(blue), y (red), and z (green) directions. The correlation lengths do not diverge,
showing that true long-range order is frustrated. See Ref. \cite{Krogstad:2019tc} for
more details.}
\end{center}
\end{figure}

For example, in sodium-intercalated V$_2$O$_5$, we used this technique to produce
$\Delta$PDFs, which eliminate any contribution from the framework V$_2$O$_5$ lattice
and leave only the sodium-sodium pair correlations \cite{Krogstad:2019tc}. The sodium
ions partially occupy sites on two-leg ladders and the 3D-$\Delta$PDF consists of
peaks at real-space vectors connecting occupied sites (Fig.
\ref{fig:V2O5-PDF-model}). The intensity of the peaks is given by the difference from
the average occupation of the conditional probability that if one site is occupied,
the connected site is also occupied. If the probability is greater (less) than in the
average structure, the $\Delta$PDF is positive (negative).  A remarkable feature of
the $\Delta$PDF is that conclusions can often be drawn by inspection without detailed
modeling. In Fig. \ref{fig:V2O5-PDF-model}, a zig-zag configuration of sodium ions
can be inferred from those interatomic vectors that have a positive $\Delta$PDF
intensity. Furthermore, these intensities decay exponentially with distance, allowing
finite correlation lengths to be determined even when the local structure has not
been solved. Fig. \ref{fig:NaxV2O5-summary} shows how the supposed order-disorder
transition in Na$_x$V$_2$O$_5$ is really a 2D to 3D crossover, which never achieves
true long-range order \cite{Krogstad:2019tc}. 

This is critical to understanding structural correlations that may not be truly
long-range. There is no other method that can yield a true thermodynamic average of
the length scale of interionic correlations in all three dimensions, both below
\textit{and} above electronic phase transitions. 

\subsubsection{Nonparametric estimation of pair distribution functions} Although it
is understood that the pair distribution function is the result of inverting the FT
on the measured intensity, difficulties arise when one considers that there is a
finite discrete set of observations, and that statistical point estimates of the pair
distribution function will thus be accompanied by some bias. Since the measured
intensity is real, positive, and even (symmetric) in $\mathbf{Q}$, its inverse FT is
real and is equivalent to its (forward) FT or cosine transform. So in this context,
the procedure of estimating the power spectrum from samples of a stationary process
is similar to the estimation of pair distribution functions except that it is the
result before squaring that is of interest. In this section, we examine some common
sources of bias through similarity with the spectrum estimation problem familiar to
the digital signal processing community. Note that the aforementioned biases are
entirely separate from error introduced by the measuring instrument.

%First, aliased peaks may confound the estimate of the PDF. In sampling theory,
%aliases can result when a process is composed of finer-scale phenomena than can be
%resolved with the chosen sampling rate. Fine-scale oscillations then become
%unidentifiable from coarser-scale oscillations. In the present context, only the
%power spectrum of the process is observable and may already contain aliasing. 

\begin{figure} 
\begin{center}
    \includegraphics[width=0.8\textwidth]{FIGS/leakage-example.pdf} 
\end{center}
\caption{$\Delta$PDF from Mo$_x$VO$_2$ transformed (a) without any taper function,
and (b) tapered with outer products of Tukey windows. Note that the outer-product
Tukey taper was effective in removing leakage artifacts parallel to the principal
axes, but was ineffective in removing artifacts along the diagonal directions.
\label{fig:tapered}} 
\end{figure}

Classical FT approaches suffer from \textit{leakage}, which we demonstrate in Figure
\ref{fig:tapered} and we now formulate mathematically. As seen in Fig.
\ref{fig:tapered}, leakage severely reduces the ability of the $\Delta$PDF method to
create an accurate image of the pair distribution function. Resolving this challenge
is the principal mission of the spectral data analysis effort of this project. 

Suppose that $x(\mathbf{s})$ denotes an unobserved $d$-dimensional stationary
zero-mean process, and $P(\mathbf{r})$ its corresponding stationary autocorrelation
function 
\begin{equation}
P(\mathbf{r}) = \mbox{E} \{ x(\mathbf{s}) x(\mathbf{s} + \mathbf{r}) \}.
\end{equation} 
where E denotes expectation. Let $S(\mathbf{Q})$ denote the spectrum of
$x(\mathbf{s})$ so that $P(\mathbf{r})$ and $S(\mathbf{Q})$ are a FT pair. In this
context $P(\mathbf{r})$ is understood to be the pair distribution function of
interest, and $S(\mathbf{Q})$ is the intensity in reciprocal space.

Suppose we observe the intensity $I(\mathbf{Q})$, which is some noisy version of $S$
on some discrete set of points $\mathcal{K}$ which forms a symmetric subset of the
hyper-rectangle $\mathcal{D} = \otimes_{j = 1}^d \frac{1}{N_j} \{\lceil -N_j/2 \rceil
+ 1, \ldots, \lfloor N_j/2 \rfloor \}$, corresponding to unit sampling in
$x(\mathbf{s})$. That is, one observes the intensity on a grid having $N_j$ points on
each of $j = 1, \ldots, p$ dimensions on which to perform a Discrete Fourier
Transform (DFT). 

We wish to form an estimate of the pair distribution function
$\widehat{P}(\mathbf{r})$ for $\mathbf{r} \in \mathcal{G}$ where $\mathcal{G} =
\otimes_{j=1}^d \{\lceil -N_j/2 \rceil + 1, \ldots, \lfloor N_j/2 \rfloor \}$. The
intensity is an even function of $\mathbf{Q}$, and we estimate $P(\mathbf{r})$ using
the DFT
\begin{align}
  \widehat{P}(\mathbf{r}) &= \frac{1}{\prod_{j=1}^d N_j} \sum_{\mathbf{Q}\in
  \mathcal{D}} U_{\mathcal{K}} (\mathbf{Q}) I(\mathbf{Q}) e^{i 2 \pi
  \mathbf{Q}^T\mathbf{r}} 
\end{align}

where $U_{\mathcal{K}}(\mathbf{Q}) = 1, \mathbf{Q} \in \mathcal{K}$ and zero
otherwise denotes the indicator function on the set $\mathcal{K}$. We see that
$\widehat{P}(\mathbf{r}) = (P*h)(\mathbf{r})$ where $*$ denotes convolution, where
the function $h(\mathbf{r})$ is defined as
\begin{equation}
 h(\mathbf{r}) = \frac{1}{\prod_{j=1}^d N_j} \sum_{\mathbf{Q}\in
  \mathcal{D}} U_{\mathcal{K}} (\mathbf{Q}) e^{i 2 \pi
  \mathbf{Q}^T\mathbf{r}}
\end{equation}

the result of which is real since $\mathcal{K}$ is a symmetric domain. In signal
processing parlance, \textit{leakage} indicates the resulting ``smearing" of the 3D
pair distribution function $P$ due to its convolution with $h$. This can be seen as a
biasing factor in the estimation of the 3D pair distribution function.

% Outlining the challenges in this area
Moreover, edge effects produce spectral leakage, since one gets discontinuities when
one periodically extends any pattern which was not both periodic in the first place
and sampled for exactly an integer number of its periods. This results in additional
spurious mass near $\mathbf{r}$'s far from the oscillation where they originated.
This smearing is a result of the intensity being reconstructed using continuous
functions when it is not perfectly periodic. Leakage can be debilitating when one has
a small sample, and since higher dimensional data has a higher proportion of the data
near an edge, leakage becomes more and more problematic with increasing dimension.
Leakage motivates the use of a \textit{data taper} \cite{brillinger1981,harris1978},
which is simply a function evaluated on the lattice which is multiplied by the
intensity. Mathematically, let $V(\mathbf{Q})$ denote the data taper, defined on
$\mathcal{D}$. By pre-multiplying the observed intensity $I(\mathbf{Q})$ with
$V(\mathbf{Q})$, the resulting PDF estimate is
\begin{align} \label{eq:Tapersneeded}
  \widehat{P}_V(\mathbf{r}) &= \frac{1}{\prod_{j=1}^d N_j} \sum_{\mathbf{Q}\in
  \mathcal{D}} U_{\mathcal{K}} (\mathbf{Q}) V(\mathbf{Q}) I(\mathbf{Q}) e^{i 2 \pi
  \mathbf{Q}^T\mathbf{r}} 
\end{align}
where  $\widehat{P}(\mathbf{r}) = (P*v)(\mathbf{r})$, and the smearing is controlled
by the properties of the function \begin{equation}
 v(\mathbf{r}) = \frac{1}{\prod_{j=1}^d N_j} \sum_{\mathbf{Q}\in
  \mathcal{D}} U_{\mathcal{K}} (\mathbf{Q}) V (\mathbf{Q}) e^{i 2 \pi
  \mathbf{Q}^T\mathbf{r}}
\end{equation}

Tapering functions frequently resemble bell shapes, which go to zero at the boundary
of $\mathcal{K}$ in an effort to smooth out the discontinuities arising near the
edges of a periodically-extended domain. This can be easily seen to reduce artifacts
in the PDFs estimated in Fig. \ref{fig:tapered}, where 3D tapers have been
constructed from 1D Tukey ones. It is essential that the function $ V(\mathbf{Q})$
behaves well on $\mathcal{K}$ and that its transform $v(\mathbf{r})$ is as similar to
a delta-function as is possible, so as to faithfully preserve $P(\mathbf{r})$. A
perfect delta-function response is not possible with a finite, discrete data
sequence, however, so one compromises by letting the tapering function be near-zero
outside a small sphere of radius $r_0$ in $\mathbf{r}$-space, while its transform is
zero outside the set $\mathcal{K}$ for which there are observations. Such optimal
tapering functions exhibit \textit{simultaneous concentration} in $\mathbf{Q}$ and
$\mathbf{r}$. In the next section we explore functions having this desirable
property.

\subsubsection{Optimal data tapers: Functions simultaneously concentrated in
$\mathbf{Q}$ and $\mathbf{r}$ \label{sec:Concentration}}

By the Paley-Wiener theorem, no function can be simultaneously $\mathbf{Q}$-limited
and $\mathbf{r}$-limited \cite{daubechies1992,hoganlakey}, that is to say, if a
function is zero outside of some discrete set of points $\mathcal{R} \subseteq
\mathbf{G} \subset \mathbb{R}^d$, its FT cannot be zero outside of $\mathcal{K}
\subseteq \mathcal{D} \subset \mathbb{R}^d$.  We consider the extent to which a
$d-$dimensional square integrable function and its FT can be simultaneously
concentrated \cite{PSWFI,PSWFV,slepian1964,simons2011}. Let $h(\mathbf{r})$ denote a
function that vanishes outside a region $\mathcal{R}$ of physical space, and let
$H(\mathbf{Q})$ denote its FT. To concentrate the energy of $h(\mathbf{r})$ into a
finite spectral region $\mathcal{K}$, we maximize the ratio
\begin{equation}
  \gamma = \frac{\int_{\mathcal{K}} |H(\mathbf{Q})|^2
  d\mathbf{Q}}{\int_{\mathbb{R}^d} |H(\mathbf{Q})|^2 d\mathbf{Q}}
\end{equation}

The optimality conditions to this problem result in the solution $h(\mathbf{r})$
satisfying the Fredholm integral equation \cite{brillinger1981} for its
eigenfunctions, or \emph{Slepian functions}, $h_0(\mathbf{r}), h_1(\mathbf{r}),
\ldots$ 
\begin{equation} \label{eq:Egval} 
  \int_{\mathcal{R}} D\left(\mathbf{r}, \mathbf{r}^{\prime}\right)
  h\left(\mathbf{r}^{\prime}\right) d \mathbf{r}^{\prime}=\gamma h(\mathbf{r}),
  \quad \mathbf{r} \in \mathcal{R}; \quad
  D\left(\mathbf{r}, \mathbf{r}^{\prime}\right)=(2 \pi)^{-2}
\int_{\mathcal{K}} e^{i \mathbf{k} \cdot\left(\mathbf{r}-\mathbf{r}^{\prime}\right)} d
\mathbf{k} 
\end{equation}
where the positive definite kernel, $D\left(\mathbf{r}, \mathbf{r}^{\prime}\right)$,
has the Hermitian property $D\left(\mathbf{r},
\mathbf{r}^{\prime}\right)=D^{*}\left(\mathbf{r}^{\prime}, \mathbf{r}\right)$.  The
solution to this eigenvalue problem yields orthogonal tapers with concentrations
$\gamma_0, \gamma_1, \ldots $ of which the first few values lie very close to one,
but then drop sharply to zero \cite{brillinger1981}. While, strictly speaking, only
the eigenvector $h_0(r)$ corresponding to $\gamma_0$ is the solution of the optimal
concentration problem, the pleasant mathematical surprise of having several
orthogonal $h_i(r)$ of almost optimal concentration strength, yet orthogonal to each
other, is used to define vastly improved estimates of the spectrum as well as
estimates of the variance (error). This is a feat that periodograms, the standard
Fourier approach, cannot accomplish. The technique is called the \textit{multitaper}
method \cite{t82}, and we will expand upon it in \S \ref{sec:MTSpectrum}.

\begin{figure}[!t]
  \begin{tabular}{cc} 
    \includegraphics[width=0.5\textwidth]{fig7a.eps} &
    \includegraphics[width=0.5\textwidth]{fig7b.eps} \\ 
%    SimpleLinefigs_pgram.eps, SimpleLinefigs_dpssK9.eps
    (a) & (b)
  \end{tabular}
  \caption{\label{fig:leakage} A 2D standing wave on a rectangular domain having 282
spatial points and a domain size of 1 spatial unit, and 306 temporal points and 1
temporal unit is generated by $x(t,s) = \sigma^2 w(t,s) + \sum_{i=1}^{\ell} A_i \cos
(2 \pi \phi_i s) \cos (2 \pi \psi_i t)$ Upper, lower panel shows periodogram (no
taper), multitaper spectra, respectively. Here $\ell = 2, A_1 = 150 = A_2$ and
$\phi_1 = 0.2, \phi_2 = 0.2$ and $\psi_1 = 0.23, \psi_2 = 0.23$, and $w(t,s)$ is unit
variance white noise with $\sigma^2 = 1$. Thus, one expects delta function impulses
at $(\pm0.2, \pm0.2)$ and $(\pm0.23,\pm0.23)$ in the spectra. Closely spaced waves
produce peaks having spurious, strong interference patterns in the periodogram, but
not in the multitaper estimate of the spectrum. This figure was reproduced from
\cite{geoga2018}.}
\end{figure}

While the vast majority of theory surrounding optimal tapering has been done for time
series analysis, \textit{i.e.}, in one dimension, our data is multidimensional. The
multidimensional tapering techniques are decidedly more limited. While outer products
of one-dimensional tapers can be used, they cannot control all leakage effects, as
can be seen in Fig. \ref{fig:leakage}. Slepian \cite{slepian1964} and Simons
\cite{simons2011} consider the two-dimensional problem in which both $\mathcal{K}$
and $\mathcal{R}$ are disk-shaped. This limitation allows a representation in polar
coordinates for which a particularly attractive form of solution exists in terms of
Bessel functions. To our knowledge, our team has been the first to use optimally
concentrated, two-dimensional Cartesian Slepian tapers in the past in fluid dynamics
\cite{geoga2018};  the goal in that work was to identify standing and traveling waves
on a 1D spatial process evolving in time. Fig. \ref{fig:leakage} shows an example of
the spectrum estimated on a superposition of two standing waves. Without tapering,
(panel (a)) power leaks from the two main peaks as cross-shaped ridges which
interfere with each other. However, after tapering with multiple disk-limited tapers,
power is confined much better to a disk with predefined width, and there is little
leakage into far-field wavenumber frequency pairs -- the positive effect of the use
of optimally concentrated tapers. 

In general, data are available on symmetric domains in $\mathbf{Q}$-space. \textit{It
is a significant focus of this project to extend the multitaper technology in
multiple dimensions} for the purpose of reducing leakage in the $\Delta$PDF analysis.
Fig. \ref{fig:tapered} demonstrates the need of true multidimensional tapering for
reducing leakage on real X-ray crystallography output. In that experiment, one uses
the Cartesian product of one-dimensional tapers, which reduces the horizontal and
vertical leakage, but power still leaks out diagonally. Slepian tapers concentrated
on the sphere will succeed in reducing the diagonal leakage as well to provide a much
clearer picture. 

It is important to note that the Slepian functions span the space of functions
simultaneously $\mathcal{K}$-limited and $\mathcal{R}$-limited. We propose that for
the purposes of PDF estimation $\mathcal{R}$ be a small spherical domain having width
$r_0$, \textit{i.e.}, $\mathcal{R} = \{\mathbf{r} \in \mathcal{G} | ||\mathbf{r}||_2
< r_0\}$ and that by solving the eigenvalue problem above, we obtain the optimal
taper to be used in \eqref{eq:Tapersneeded}. The compromise in selecting $r_0$ is
that encountered in a closely related problem in nonparametric spectrum analysis. As
is shown in \cite{simons2011}, the smaller the sphere in $\mathbf{r}$-space the fewer
large values for the concentration that can be obtained. 

\textit{The problem of computation of the concentrated tapers will constitute a
significant part of our effort}. To solve the eigenvalue problem \eqref{eq:Egval} it
is necessary to first discretize the integral equation on a large, three-dimensional
discrete domain through quadrature \cite{press1992}. The problem then becomes a
matter of solving a large matrix eigenvalue equation, for which the matrix is dense.
Using techniques from integral equations, however, the resulting problem can be
efficiently and accurately solved, using the Nyström approximations
\cite{nystrom1930,atkinson2005theoretical}, that employ low-rank approximations to
the matrix. We have recently used to great effect a hierarchical off-diagonal
low-rank version of this idea for very large-scale Gaussian process analysis
\cite{geoga2018scalable}, which can accommodate very peaked kernels in
\eqref{eq:Egval}  for which low-rank approximations may not be suitable, though we
still refer to this class of techniques as "Nyström approximations."

In three dimensions, the problem size quickly becomes very large, and requires
advanced mathematical numerical linear algebra techniques for their efficient
solution. When only the zeroth order Slepian function is desired, power iteration can
be used to obtain it in O($N^2$) cost per step where $N = \prod_{k = 1}^p N_k$.
However, when $J$ eigenvalue/eigenvector pairs are required, the problem complexity
increases significantly. On the other hand, computing a prescribed set of
eigenvalue-eigenvector pairs can be carried out with Lanczos iterations, which can be
executed scalably once matrix-vector multiplications are provided by means of Nyström
approaches using, for example, the ARPACK package \cite{lehoucq1998arpack}. 

In the next section, we describe a closely related spectrum estimation problem which
utilizes the first $J$ tapers having large eigenvalues. This method is known as the
multitaper method in the literature \cite{t82} and has numerous advantages over
estimates of the spectrum using only a single taper. Most notably, it is possible to
not only estimate the power spectrum with high accuracy but to provide estimates of
its variability. Computation of three-dimensional Slepian tapers would make possible
high-accuracy spectral analysis in three or more dimensions -- a novel avenue for
research in signal processing that we believe will tremendously improve the
$\Delta$PDF methods, but which is also likely to positively impact a myriad of other
applications, mostly in geophysics and seismology. 

\subsubsection{Spectrum and Cepstrum Analysis in $d-$dimensions \label{sec:MTSpectrum}}

We concentrate momentarily on the closely related problem of estimating the
nonparametric spectral density of a multidimensional process. Denoting by
$v^{(j)}(\mathbf{r})$, $j = 0,\ldots, J-1$, a set of $J$ tapers evaluated on
$\mathcal{D} \subset \mathbb{Z}^2$, that solve the discrete analog of
\eqref{eq:Egval} described in \cite{slepian1964}, one computes the multitaper
spectral estimator \cite{t82}, denoted $\hat{S}(\mathbf{Q}) $, by averaging $J$
independent estimates of the power spectrum \begin{equation} \label{eq:mt}
\hat{S}^{(j)}(\mathbf{Q}) = \frac{1}{\prod_{j=1}^p N_j} \sum_{\mathbf{r} \in
\mathcal{D}} v^{(j)}(\mathbf{r}) x(\mathbf{r}) e^{-2\pi i \mathbf{Q}^T \mathbf{r}};
\quad \hat{S}(\mathbf{Q}) = \frac{1}{J} \sum_{j=0}^{J-1} \hat{S}^{(j)}(\mathbf{Q}).
\end{equation} The essential observation is that, due to the orthogonality of the
tapers, stemming from their being eigenvectors for different eigenvalues of a
Hermitian operator, and due to their optimal concentration property (almost exact for
all $J$, as we discussed above), as well as asymptotic normality of spectra of
stationary processes, we can interpret \eqref{eq:mt} as a sample average of
identically distributed and independent random variables. 

Therefore, the tapers themselves control the leakage, or bias, of the estimator,
while the averaging operation reduces the variability. In fact, \textit{there is no
estimator that can better eliminate spectral leakage on 3D data by construction}
\cite{thomson2012}. As mentioned earlier, adjustment of the radius of the region over
which the tapers are concentrated allows one to obtain more or fewer tapering
functions. This means that one can obtain less bias at the expense of more variance
\cite{percival1993,bronez92} and \textit{vice versa} by adjustment of this parameter.

The ability to design optimal tapers concentrated on the domain for which there are
observations (arbitrary shapes and not simply on a region shaped like a hypercube),
while reliably being concentrated on a small disk of predefined radius in spectral
space, has unprecedented power to analyze Cartesian data from all sorts of measuring
instruments. The multitaper method outperforms other nonparametric spectral
estimators on conventional univariate spectrum analysis problems, by minimizing
spurious detections of pure oscillatory components \cite{th14}. Under certain
conditions, it is also the maximum likelihood estimator for the spectrum,
\cite{stoica99}. 

Briefly, one can estimate the variability of the multitaper spectrum estimate by
resampling methods. By computing the sample variance of the values that result when
one systematically leaves out one of the independent estimates ("jackknife"
re-sampling \cite{efron1981nonparametric}) $\widehat{S}^{(j)}(\mathbf{Q})$ above, one
can get a remarkably useful measure of the error in the estimate, a property that we
believe can be valuable in the present application as well.

 The multitaper approach also allows us to compute bootstrapped confidence intervals,
using the leave-one-out method just described. In particular, this will provide us
with error estimates of the pair distribution function. It will allow us to
discriminate between different structural signature explanations to the given diffuse
diffraction pattern. 

We will investigate multidimensional multitapering approaches for dramatically
reducing leakage in the $\Delta$PDF method and provide it with error estimates. While
the initial spatial concentration domain we will attempt will be spherical, existence
of anisotropy may lead us to investigate other shapes as well. An important issue is
investigating the optimal bias between bias and variability. Therefore, we aim to
choose the concentration radius $r_0$ using an optimal mean squared error criterion,
an issue we have resolved in one dimension by identifying the asymptotic dependence
of the taper bias on the spectrum \cite{ha2017}. 

% \subsubsection{Cepstrum analysis \label{sec:Cepstrum}}

In the field of digital signal processing, the cepstrum \cite{childers1977}, is
defined as the inverse FT of the logarithm of the power spectrum of a stationary
process $x(\mathbf{s})$. The cepstrum may provide additional insight into the
regular, periodic nature of the intensity signal. The cepstrum treats the spectrum as
if it were a stationary process and attempts to fit a cosine series to its logarithm.
It is mainly used in speech processing to measure vocal pitch, for example, where the
spectrum has a more slowly-varying component with regularly spaced harmonics of a
fundamental frequency superimposed. The pitch is the spacing between these peaks, and
the cepstrum has a single large peak representing the regular, large peaks in the
spectrum. It is additionally well suited to represent regular echoes with damping.
The cepstrum is indexed in the original space, but indices are called quefrency to
keep track of the logarithmic conversion \cite{hansson2009}. 

The logarithmic transformation suppresses amplitude differences in processes having
large spectral range. In this context, Bragg peaks overpower the diffuse part of the
scattering by several orders of magnitude, and a representation of these low-power
features might be artificially possible even before the use of the punch-and-fill
method by means of estimating the spectral envelope. That is, Bragg peaks, or any
phenomenon with regular periodicity in intensity will be compactly represented in
quefrency by a single, large cepstrum coefficient and can be filtered from the
result. In this context, we expect that the cepstrum may be useful as an alternative
real-space representation for the regular part of the sample. 

\subsubsection{Physical interpretation of structural signatures \label{sec:Impact}}

From the clusters identified using ML, one would like to isolate and analyze these
features in both reciprocal and real space to interpret their physical significance.
When 3D PDF is analyzed with both Bragg peaks and diffuse scattering included, the
origin of the correlation space becomes cluttered, and it is difficult to separate
individual phenomena. 

As we have outlined in section \ref{sec:Concentration}, the Slepian functions are a
flexible vehicle for isolating and shuttling power from arbitrary regions of
reciprocal to real space. Thus, they can be customized to convert
temperature-dependent signatures of transition phenomena in intensity to correlation
space, where they have physical significance in terms of interatomic vector
correlations. In particular, we envision that ML approaches in section \ref{sec:ml}
can classify temperature-dependent diffuse scattering patterns in reciprocal space,
and by solving the concentration problem over the pattern identified by
classification, we can faithfully estimate their real-space analogues, even in the
presence of nearby structure with large dynamic range. 

%Slepian tapers are not the only functions that can be used as multidimensional
%filters, but linear combinations of these can be used to create so-called projection
%filters, which produce zero phase distortion. 

% \subsubsection{Data Analysis and Signal Processing Impact}

% Significance and novelty

The research we propose aims to vastly expand our knowledge of multidimensional
spectrum estimation and bring optimal-concentration-driven advances to significantly
increase the accuracy of the $\Delta$PDF method and accelerate the discoveries it
will prompt. We believe our team will be the first group to ever try such methods in
materials science/crystallography applications. 

Specifically, to our knowledge, there is no 3D spectrum estimation on Cartesian
geometries being performed in the literature, and the design of optimal data tapers
has been described \cite{slepian1964,simons2011,hoganlakey} but these have not been
computed in three dimensions to be applied to real data.  The main reasons for this
may be that real data of dimension larger than two is rarely available on a regular
lattice, even when it is the output of a computer simulation \cite{geoga2018}.  In
fact, the main application of optimal higher-dimensional tapers to spectrum
estimation has been in the geophysical literature \cite{kirby2014}, most often on the
sphere \cite{simons2011}. 

The spectral analysis of high-dimensional signals is one that is rarely broached in
the signal processing community, but thanks to the availability of high-dimensional
datasets with regular sampling, such as those produced in this proposal, it is likely
to become commonplace. We have shown that the statistical properties of tapering
produce a far more accurate spectrum and PDF estimate, both in bias and variance of
the estimate, and we have outlined the design of novel three-dimensional tapers to be
used for $\Delta$PDF estimation with minimal bias and estimation artifacts. The
computation of optimal Slepian functions in three dimensions to mitigate leakage
would represent a formidable novel contribution to the field of statistics and signal
processing. 

